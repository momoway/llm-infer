% TEMPLATE for Usenix papers, specifically to meet requirements of
%  USENIX '05
% originally a template for producing IEEE-format articles using LaTeX.
%   written by Matthew Ward, CS Department, Worcester Polytechnic Institute.
% adapted by David Beazley for his excellent SWIG paper in Proceedings,
%   Tcl 96
% turned into a smartass generic template by De Clarke, with thanks to
%   both the above pioneers
% use at your own risk.  Complaints to /dev/null.
% make it two column with no page numbering, default is 10 point
% Munged by Fred Douglis <douglis@research.att.com> 10/97 to separate
% the .sty file from the LaTeX source template, so that people can
% more easily include the .sty file into an existing document.  Also
% changed to more closely follow the style guidelines as represented
% by the Word sample file. 
% Note that since 2010, USENIX does not require endnotes. If you want
% foot of page notes, don't include the endnotes package in the 
% usepackage command, below.
\documentclass[letterpaper,twocolumn,9pt]{article}
\usepackage{usenix,epsfig,endnotes,xspace,listings}
\usepackage{booktabs}
\usepackage{adjustbox}
\usepackage{multirow}
\usepackage{caption}
\usepackage{tabularx}
\usepackage{placeins}
\usepackage{minted}
\lstset{
    numbers=left, % 显示行号
    numberstyle=\tiny, % 行号字体
    keywordstyle=\color{blue!70}, % 关键字颜色
    commentstyle=\color{red!50!green!50!blue!50}, % 注释颜色
    frame=shadowbox, % 为代码块添加阴影框
    rulesepcolor=\color{red!20!green!20!blue!20}, % 阴影框颜色
    xleftmargin=2em, xrightmargin=2em, aboveskip=1em, % 设置代码块的边距
    framexleftmargin=2em % 阴影框左边距
} 
\usepackage[T1]{fontenc}
\usepackage[scaled]{berasans}
% 设置等宽字体为 Consolas 风格
\renewcommand*\ttdefault{txtt}
\newcommand{{\cfos}}{CFOS\xspace}
\begin{document}
%don't want date printed
\date{}
%make title bold and 14 pt font (Latex default is non-bold, 16 pt)
\title{\Large \bf INDENG 174 Group 9 Project Progress Report}
\author{
{\rm Runyuan He} \\
{\rm 3041920716}
\and
{\rm Jiedong Zhang} \\
{\rm 3041913865}
\and
{\rm Qingyang Xu} \\
{\rm 3041979645}
}
\maketitle
% Use the following at camera-ready time to suppress page numbers.
% Comment it out when you first submit the paper for review.
\thispagestyle{empty}
\section{Problem Definition and Research Questions}

Building upon our initial proposal, we have refined our focus to address specific performance bottlenecks in LLM inference systems. Our simulation targets the complex interplay between batching strategies and scheduling policies in systems like vLLM \cite{kwon2023efficient} and SGLang \cite{zheng2024sglangefficientexecutionstructured}. Following are some specific problems we want to solve.

\begin{itemize}
\item \textbf{Dynamic Batch Size Adaptation} We are investigating how to dynamically adjust batch sizes based on real-time queue lengths and arrival patterns. The key challenge is determining when the throughput gains from larger batches outweigh the increased queueing delays.

\item \textbf{Heterogeneous Request Handling} With requests varying from 10 to 2000+ tokens in length, we need to understand how different scheduling policies (FCFS, SJF, priority-based) handle this heterogeneity. Our focus is on minimizing p99 latency while maintaining high throughput.

\item \textbf{Non-stationary Load Patterns} Real-world systems experience time-varying loads (e.g., 10x spikes during peak hours). We are modeling how different batching strategies perform under these non-homogeneous Poisson arrivals with rates $\lambda(t)$ varying from 2 to 20 requests/second.
\end{itemize}

\input{tex/current_progress.tex}
\input{tex/experiments.tex}
\bibliographystyle{acm}
\bibliography{reference}
\clearpage
\end{document}